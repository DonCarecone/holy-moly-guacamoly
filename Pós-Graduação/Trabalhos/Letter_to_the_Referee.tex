\documentclass[11pt,superscriptaddress,aps,prd,preprint]{revtex4-2}
%\usepackage{amsmath}

%\usepackage{amssymb}
\makeatletter
%\usepackage{babel}

%\usepackage[brazilian]{babel}
%\usepackage[utf8]{inputenc} 
\usepackage[T1]{fontenc}
\usepackage{amsmath}
\usepackage{amssymb}
\usepackage{graphicx}

\usepackage{fixmath}
\everymath{\displaystyle}
\usepackage{graphicx}


\usepackage{ifpdf}
\usepackage{hyperref}
\usepackage{xcolor,color,graphicx,graphics,physics}
\usepackage[spanish,english]{babel}%, portuguese
%\usepackage[latin1]{inputenc}
\usepackage[OT1]{fontenc}
\usepackage{latexsym,amssymb,amsmath,amsfonts, slashed,cancel}
\usepackage{makeidx}
\usepackage{epsfig,subfigure}
\usepackage{natbib}
\usepackage{epstopdf}
\usepackage{mathrsfs}
\usepackage{hyperref}%\usepackage[colorlinks=true,linkcolor=blue]{hyperref}
\hypersetup{colorlinks=true, linkcolor=blue, citecolor=blue, urlcolor=blue}
\usepackage{enumerate}


\usepackage{slashed}
\def\lrprtmu{\stackrel{\leftrightarrow}{\partial_\mu}}
\def\lrprt0{\stackrel{\leftrightarrow}{\partial^0}}
\def\lrprtj{\stackrel{\leftrightarrow}{\partial^j}}


\newcommand{\bea}{\begin{eqnarray}}
\newcommand{\eea}{\end{eqnarray}}
\newcommand{\vk}{\vec{k}}
\newcommand{\e}{\epsilon}
\newcommand{\pa}{\partial}

% Optional packages
%\usepackage[active]{srcltx}
%\usepackage{hyperref}

\usepackage{xcolor}



\begin{document}

\begin{picture}(370,55)(0,0)
\setlength{\unitlength}{0.6pt}
\put(15,10){\includegraphics[width=3cm]{logoufmt}}
%\put(620,25){\includegraphics[width=3cm]{logobach.jpeg}}
\put(420,110){ Universidade Federal de Mato Grosso}
\put(420,75){  Cuiab\'{a} - MT, Brazil}
\put(420,40){ December 18, 2024.}
%\put(180,20){\large    Aluno:}
\put(-15,2){\line(1,0){800}} %coloca uma linha na horizontal
\put(-15,-2){\line(1,0){800}}
\end{picture}




\vspace{0.3cm}

\begin{flushleft}
{ \large The European Physical Journal C}
\end{flushleft}

\vspace{0.3cm}

Dear Editor,

\vspace{0.3cm}


I am writing this letter, on behalf of my colleagues, in response to comments made by the Referee about our article ``Gravitational Compton scattering at zero and finite temperature'', submitted under ID  EPJC24-AR-07-176.

First, we would like to thank the Referee for the comments and suggestions to improve the quality of the manuscript.
Accordingly, we have made modifications (highlighted in red) to clarify the criticism that the referee mentioned in his report.

\vspace{0.3cm}

{\bf List of changes and answers to the Referee's Report}

\begin{itemize}
	\item Reviewer:
\end{itemize}




{\it In a previous publication (Eur.Phys.J.C 83 (2023) 1, 25  e-Print: 2212.13820
	[hep-th]), which for some reason is not referenced here, one of the authors
	looked at Compton scattering at zero and finite temperature and pointed
	out that since there are lots of photons and matter in the universe this is a
	relevant thing to calculate. The present submission is similar but notes that
	there are also lots of gravitons and looks at gravitational Compton scattering
	in a similar fashion.}


\vspace{0.3cm}

{\bf Answer:} We are very grateful for the referee's comments and fully agree with the suggestion. The mentioned publication is well-known to us, and its absence in our manuscript was an oversight. In light of this, the referenced work has now been properly cited in both the introduction and the final discussion of the manuscript. Furthermore, we agree with the referee's observation regarding the relevance of this study. Considering that, in theory, the universe contains a vast number of gravitons and that we have a framework describing the interaction between gravitons and fermions, it provides a compelling motivation for investigating such interactions within the context of Compton scattering.  

\vspace{0.3cm}

\begin{itemize}
\item Reviewer:
\end{itemize}




{\it This is in principle a useful thing to do, but I have problems with two aspects of the manuscript. The most important is that the authors utilize the
	graviton-matter interaction in the GEM formalism, whereas it is generally
	acknowledged that such interactions are instead correctly described by general relativity (GR). Indeed perturbative GR is well established and has been
	described by Bjorken as the most successful effective field theory because of
	its extraordinarily wide range of applicability. I presume that the reason
	that the authors eschew GR is that it requires four diagrams at tree level
	(including one requiring the dreaded triple-graviton vertex) rather than the
	two used in GEM. However, any realistic calculation must use GR. Actually,
	at zero temperature the double copy theorem means that the GR gravitational Compton amplitude is given in terms of its Compton analogs, so that
	tremendously simplifies things. However, I’m skeptical that the double copy
	remains valid at finite temperature.}


\vspace{0.3cm}

{\bf Answer:} Thank you for the suggestions and comments. We fully agree with the referee's observation. Perturbative GR is a well-established effective theory with a broad range of applications. However, our choice to use GEM for the calculation of gravitational Compton scattering stems from its analogy with QED, which enables a direct comparison between the scattering processes in the two theories and provides a clear and intuitive description of the graviton in a manner similar to the photon. It is important to emphasize that GEM does not neglect GR. On the contrary, many physical processes can be described using either theory. Nonetheless, GEM's Lagrangian formulation, along with its effective description of graviton interactions with fermions and photons, makes it a more suitable framework for the study of scattering amplitudes and cross sections in quantum field theory. In response to the referee's comment, we have included a more detailed discussion of GEM, its similarities with GR and QED, a comparison of Compton scattering in the GEM and QED formalisms, and an expanded set of references that compare these theories, including the one suggested by the referee, in the revised manuscript. These changes have been incorporated into the introduction (Section I), at the beginning of Section II and in the Conclusions. 


\vspace{0.3cm}
\begin{itemize}
	\item Reviewer:
\end{itemize}

\vspace{0.3cm}

{\it The second aspect which I have difficulty with is the casual use of TFD in
	order to treat finite temperature effects. This formalism is not the standard
	(closed time path) one and in fact yields different results, as shown in the
	Compton paper referenced above. Because of this readers should be carefully
	introduced to this method and a strong justification needs to be made for its
	use.}

\vspace{0.3cm}

{\bf Answer:} We are very grateful for the referee's suggestion. In fact, in addition to TFD, there are several formalisms for introducing thermal effects in quantum field theory, such as Matsubara and CTP. The reason we chose TFD in our manuscript is that it provides an effective description of scattering processes occurring at tree-level and in thermal equilibrium. On the other hand, CTP, although capable of describing both equilibrium and non-equilibrium situations, is a formalism more commonly applied to non-equilibrium physics. While TFD may not be as widely known for introducing temperature, it is well-supported by a variety of works that describe its framework and applications to physical processes. With this in mind, we have added a more extensive description of the formalism, as well as detailed explanations of concepts that may cause confusion for readers, in the revised manuscript. These changes have been incorporated into Section I, Section IV, and the Conclusions.

\vspace{0.3cm}


\begin{flushleft}
{Sincerely,}\\
\vspace{0.2cm}
Prof. Alesandro F. Santos
\end{flushleft}



\end{document}
